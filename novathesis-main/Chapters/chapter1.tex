%!TEX root = ../template.tex
%%%%%%%%%%%%%%%%%%%%%%%%%%%%%%%%%%%%%%%%%%%%%%%%%%%%%%%%%%%%%%%%%%%
%% chapter1.tex
%% NOVA thesis document file
%%
%% Chapter with introduction
%%%%%%%%%%%%%%%%%%%%%%%%%%%%%%%%%%%%%%%%%%%%%%%%%%%%%%%%%%%%%%%%%%%

\typeout{NT FILE chapter1.tex}%

\chapter{Introduction}
\label{cha:introduction}

\prependtographicspath{{Chapters/Figures/Covers/}}

% epigraph configuration
\epigraphfontsize{\small\itshape}
\setlength\epigraphwidth{12.5cm}
\setlength\epigraphrule{0pt}

In this chapter, the context of the dissertation is established with respect to previously developed related work, thereby outlining the motivation for the topic and the intended objectives and the main contributions.. A concise overview of the dissertation’s structure is also provided.

\section{Context and Motivation of the Work}
\label{sec:context_and_motivation_of_the_work}


The contemporary engineering landscape is undergoing a profound transformation, often termed the Fourth Industrial Revolution or Industry 4.0. This paradigm is characterized by the deep integration of cyber-physical systems (CPS), the Internet of Things (IoT), and cloud computing into industrial processes~\cite{Kagermann2013}. At the core of this evolution lies a fundamental shift from centralized monolithic control architectures towards highly distributed and interconnected embedded systems~\cite{Lee2017}. These systems are composed of numerous intelligent nodes, sensors, actuators, and controllers, that must cooperate in a coordinated manner to achieve complex tasks, from smart manufacturing to autonomous logistics.

This distribution offers significant advantages, such as improved modularity, scalability, and fault tolerance. However, it also introduces unprecedented levels of complexity in system design, verification, and real-time coordination~\cite{AlFuqaha2015}. Managing the concurrency, synchronization, and communication inherent in these systems requires formal modeling techniques capable of capturing their dynamic behavior with mathematical rigor. It is within this modern context that Petri nets, a formalism with a long-standing history, find renewed and critical relevance.

The concept of Petri nets was first introduced by Carl Adam Petri in his 1962 dissertation "Kommunikation mit Automaten"~\cite{petri1962}. Originally proposed as a modeling technique for distributed and concurrent systems in computer science, Petri nets have since become widely adopted in the fields of embedded systems and software development.

Petri nets provide a robust and intuitive framework for modeling and analysis of complex systems. Their graphical representation facilitates the visualization of concurrent, asynchronous, and distributed processes, while their underlying mathematical formalism enables rigorous validation and verification. Due to their versatility and expressive power, Petri nets have proven to be a valuable tool for both theoretical research and practical engineering applications~\cite{murata}.

Among the numerous tools available for Petri net modeling, the \gls{iopt}-Tools environment underpins a specialized approach to developing controllers through \gls{iopt} models, a specific class of Petri nets. The \gls{iopt}-Tools framework, accessible at \url{http://gres.uninova.pt/IOPTTools/}, provides mechanisms for decomposing complex models into independent sub-models, allowing them to execute across distinct computational nodes.

The increasing complexity of distributed control system models, particularly under the \gls{gals} paradigm, has intensified the need for efficient and dependable communication between these distributed sub-models. As systems become more modular and geographically distributed, ensuring low-latency data exchange and maintaining operational consistency pose significant challenges. These challenges are particularly acute in the context of distributed \gls{iopt}sub-models and \gls{gals}, as manual communication handling can lead to extensive development effort, introduce the potential for errors, and complicate the formal verification of system behavior.

\section{Problem Statement}
\label{sec:problem_statement}


At the heart of this work lies the challenge of integrating effective communication channels into the \gls{iopt}-Tools environment. While \gls{iopt}-Tools excels at defining individual model logic and decomposing complex systems, its current capabilities do not extend to the automated generation of model communication infrastructure. 
This absence of automation relegates the implementation of inter-controller communication to a manual, ad-hoc process. Such a process is not merely inefficient; it is fraught with significant risks that can undermine the reliability and economic viability of the system. Manual coding of communication protocols is inherently error-prone, increasing the likelihood of subtle implementation defects, such as race conditions and deadlocks, which are notoriously difficult to detect and debug in a distributed environment~\cite{Broy2012}. Furthermore, this approach elevates development costs and extends the time-to-market, which are critical concerns in competitive industrial settings~\cite{Stahl2006}. It also places a heavy cognitive burden on system designers, requiring them to possess deep expertise in both high-level control modeling and the low-level intricacies of multiple hardware and network protocols, which complicates system maintenance and hinders the reusability of the resulting code~\cite{Sommerville2011}.

The main problem is to establish a robust mechanism that enables the distributed controllers, each executing an independent \gls{iopt}sub-model, to exchange data efficiently. This is complicated by the variety of communication technologies available, each with its distinct advantages and constraints, which raises the question of how best to manage this diversity and meet the rigorous performance requirements of distributed automated systems.

\section{Objectives}
\label{sec:objectives}

This project is structured around two main objectives. The first is a comparative analysis of communication technologies, encompassing both wired point-to-point networks such as \gls{i2c}, \gls{spi}, and \gls{uart} and wireless solutions including Wi-Fi and Bluetooth. These technologies are evaluated against criteria such as simplicity, reliability, and latency in order to identify their suitability for distributed embedded systems. The second objective is the development of an automated code generation tool capable of producing the communication modules required for each \gls{iopt} sub-model. By automating this process, the tool ensures efficient data exchange between controllers while maintaining robustness and responsiveness within the overall system.

The successful fulfillment of these objectives leads to four primary contributions. First, a functional web-based \gls{api} was designed and implemented to automate the generation of communication modules for three widely used protocols (\gls{i2c}, \gls{uart}, and \gls{tcp}/\gls{mqtt}), directly addressing the communication gap in the \gls{iopt}-Tools framework. Second, the research validates an end-to-end model-driven workflow, spanning from high-level \gls{iopt} Petri net specifications to a deployed multi-controller hardware implementation, as demonstrated through a comprehensive real-world case study. Third, an empirical performance analysis of the generated modules quantifies memory footprint and communication latency, providing a data-driven basis for selecting communication technologies in resource-constrained systems. Finally, the developed \gls{api} has been publicly released as an open-source artifact, thereby promoting transparency, enabling reproducibility, and offering a foundation for future community-driven extensions. To further support reproducibility and long-term development, the software is available in a dedicated source code repository.


\section{Dissertation Structure}
\label{sec:dissertation_structure}


This dissertation is organized into five main chapters, each addressing a key component of the research:

\begin{itemize}
    \item \textbf{Chapter 1 - Introduction:} This chapter introduces the context and motivation for the research, defines the problem statement, presents the main objectives, and outlines the structure of the document.

    \item \textbf{Chapter 2 - State of the Art:} This chapter provides a thorough review of the theoretical and technological foundations for the work, covering Petri nets, the \gls{iopt}formalism, \gls{gals} architectures, and a detailed analysis of relevant communication protocols.

    \item \textbf{Chapter 3 - System Design and Implementation:} This chapter details the architecture and design of the automated code generation tool. It presents the rationale for the design, the mapping from model constructs to code, the formal \gls{api} specification, an analysis of the generated code, and the methodology for the tool's validation.

    \item \textbf{Chapter 4 - Case Study and Performance Analysis:} This chapter presents a comprehensive case study applying the developed tool to a real-world multi-controller automation system. Details the implementation process and provides an empirical performance analysis of the generated communication modules, focusing on memory footprint and dynamic behavior.

    \item \textbf{Chapter 5 - Conclusion:} The final chapter summarizes the dissertation, reiterates the main contributions, discusses the limitations of the current work, and proposes concrete directions for future research.
\end{itemize}


