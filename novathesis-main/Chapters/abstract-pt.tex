%!TEX root = ../template.tex
%%%%%%%%%%%%%%%%%%%%%%%%%%%%%%%%%%%%%%%%%%%%%%%%%%%%%%%%%%%%%%%%%%%%
%% abstract-pt.tex
%% NOVA thesis document file
%%
%% Abstract in Portuguese
%%%%%%%%%%%%%%%%%%%%%%%%%%%%%%%%%%%%%%%%%%%%%%%%%%%%%%%%%%%%%%%%%%%%

\typeout{NT FILE abstract-pt.tex}%


O desenvolvimento orientado a modelos, particularmente com o uso de formalismos como as Redes de Petri Input-Output Place-Transition (IOPT), oferece uma abordagem robusta para o projeto de controladores embebidos complexos. A plataforma IOPT-Tools suporta este paradigma, automatizando a geração da lógica dos controladores, mas existe uma lacuna significativa na sua capacidade de gerar automaticamente a infraestrutura de comunicação necessária para sistemas distribuídos. Isto exige um processo manual, demorado e propenso a erros para implementar as ligações de comunicação entre controladores.

Esta dissertação aborda esta lacuna, apresentando o desenho, a implementação e a validação de uma Application Programming Interface (API) baseada na web para a geração automatizada de módulos de comunicação. A ferramenta produz dinamicamente código C++ para três protocolos distintos, I²C, UART e TCP/MQTT,  projetado para integração direta no fluxo de trabalho das IOPT-Tools.

A utilidade e o desempenho do código gerado foram validados através de um estudo de caso abrangente de um sistema distribuído de quatro controladores para três tapetes rolantes. Uma análise empírica forneceu métricas de desempenho quantitativas, demonstrando os significativos compromissos (\textit{trade-offs}) entre os protocolos: protocolos de hardware como o I²C (latência de 185 µs, consumo de memória residual) ofereceram um desempenho superior em tempo real, enquanto o protocolo de rede TCP/MQTT (latência de 45 ms, consumo superior a 26 kB) proporcionou maior flexibilidade funcional. O comportamento dinâmico do sistema implementado foi também verificado através de diagramas temporais gerados pelo simulador.

Os resultados confirmam que a ferramenta desenvolvida acelera com sucesso o desenvolvimento de sistemas distribuídos, reduz os erros de implementação e capacita os projetistas com os dados empíricos necessários para tomar decisões informadas e baseadas em dados, ao equilibrar os requisitos de desempenho com as restrições de recursos dos sistemas embebidos.
